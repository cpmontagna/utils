\documentclass[fleqn,11pt]{article}
\setlength{\parindent}{0em}

\usepackage[normalem]{ulem} % provides underlining
\usepackage[dvips]{graphics}
\usepackage[dvips]{graphicx}
\usepackage{epsfig}
\usepackage{listings}
\usepackage{booktabs}
%\usepackage{threeparttable}
\usepackage{multirow}
\usepackage{longtable}
\usepackage[usenames]{color}
\usepackage{rotating}
%\usepackage[italian]{babel}
\usepackage[merge]{natbib}  % merge to use pre and post citation keys
\usepackage{amsmath}
\usepackage{amsfonts}
\usepackage{amssymb}
\usepackage[cdot,amssymb]{SIunits}
\usepackage[detect-all]{siunitx}    % better than the above
\usepackage{ctable}
\usepackage{enumerate}
\usepackage{cancel}
\usepackage{esint}   % alternate integral symbols
\usepackage{polynom}
\usepackage{wasysym} % symbols
\usepackage{chngcntr} % change counter resetting
\usepackage{authblk}

%\definecolor{darkgreen}{rgb}{0, .6, 0}

% Cauchy principal value integral
\newlength{\intwidth}
\DeclareRobustCommand{\fpint}[2]
   {\mathop{%
      \text{%
        \settowidth{\intwidth}{$\int$}%
        \makebox[0pt][l]{\makebox[\intwidth]{$-$}}%
        $\int_{#1}^{#2}$}}}
 
\begin{document}
\lstset{language=MATLAB,basicstyle=\small\ttfamily}
\bibliographystyle{plainnat} 
\bibpunct{(}{)}{;}{a}{,}{,}

\title{\huge\bf DI events}
\author{Chiara P. Montagna}
\affil{Istituto Nazionale di Geofisica e Vulcanologia, Pisa}
\date{\today}
\maketitle

\begin{figure}[h]
  \centering
  \includegraphics[width=\textwidth]{setup+different_geometries.eps} % needs \usepackage{graphicx}
  \caption{Initial conditions for the numerical simulations of the
    magmatic system. On the left, the whole domain is showm,
    indicating the two magmatic end-members. On the right, the upper
    portion of the domain shows the three different geometries
    explored.}
  \label{setup+different_geometries}
\end{figure}

\bibliography{/home/montagna/TOURO/science/bibliography/biblio} 
\end{document} 