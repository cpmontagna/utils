\documentclass[fleqn,11pt]{article}
\setlength{\parindent}{0em}

\usepackage[exponent-product=\cdot]{siunitx}
\usepackage{natbib}
\usepackage{authblk}  % for affiliations
\usepackage{listings}
\usepackage[usenames]{color}
\usepackage[T1]{fontenc}
\usepackage[Q=yes]{examplep}
\usepackage{booktabs} % linee orizzontali in tabelle
\usepackage{ctable}   % tabelle con note, includes booktabs
\usepackage[usenames,dvipsnames]{pstricks}  % latexdraw
\usepackage{epsfig}                         %   
\usepackage{pst-grad} % For gradients       %
\usepackage{pst-plot} % For axes            % latexdraw
\usepackage{graphicx}
\graphicspath{{magmaFoam/BubbleRP/}{figures}} % where to look for figures
\usepackage{import}   % when inputting external files, use relative
                      % path for nested inclusions
\usepackage{amsthm}   % theorems
\usepackage{amsmath}
\usepackage[makeroom]{cancel}   % crossed-out maths
\usepackage{tikz}

\makeatletter

% flush right within align environment
\newcommand{\pushright}[1]{\ifmeasuring@#1\else\omit\hfill$\displaystyle#1$\fi\ignorespaces}
\makeatother

% more than one programming language with listings:
\lstnewenvironment{CPP}
  {\lstset{language=C++,basicstyle=\ttfamily\small,frame=none}}
  {}

\lstnewenvironment{matlab}
  {\lstset{language=MATLAB,basicstyle=\small\ttfamily}}
  {}
% use as \begin{CPP} ... \end{cpp}

\begin{document}
\bibliographystyle{plainnat} 
\bibpunct{(}{)}{;}{a}{,}{,}
\lstset{language=MATLAB,basicstyle=\small\ttfamily}

\title{}
\author{Chiara P. Montagna}
\affil{Istituto Nazionale di Geofisica e Vulcanologia, Pisa}
\date{\today}
\maketitle

\setcounter{tocdepth}{2} % depth up to which contents are listed in toc
\tableofcontents

\input{/home/montagna/elastic_dikes/kilauea/DI_events/plan_input.tex}

\begin{figure}[h]
  \centering
  \includegraphics[width=\textwidth]{setup+different_geometries.eps} % needs \usepackage{graphicx}
  \caption{Initial conditions for the numerical simulations of the
    magmatic system. On the left, the whole domain is showm,
    indicating the two magmatic end-members. On the right, the upper
    portion of the domain shows the three different geometries
    explored.}
  \label{setup+different_geometries}
\end{figure}

% \section{Track-keeping}
% \begin{itemize}
% \item [Jan 3, 2014] 
% \end{itemize}

% \section{TODO}
% \begin{enumerate}
% \item
% \end{enumerate}

\bibliography{/home/montagna/TOURO/science/bibliography/biblio} 
\end{document}
