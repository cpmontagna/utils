\documentclass[a4paper,notitlepage]{article}
\setlength{\parindent}{0em}

\usepackage{SIunits}
% \usepackage{natbib}
% \usepackage{authblk}
% \usepackage[usenames]{color}
% \usepackage[T1]{fontenc}
% \usepackage[Q=yes]{examplep}
% \usepackage{booktabs} % linee orizzontali in tabelle
% \usepackage{ctable}   % tabelle con note, includes booktabs
% \usepackage[usenames,dvipsnames]{pstricks}  % latexdraw
% \usepackage{epsfig}                         %   
% \usepackage{pst-grad} % For gradients       %
% \usepackage{pst-plot} % For axes            % latexdraw
% \usepackage{graphicx}
% \usepackage{import}   % to import svg files from another folder
% \usepackage{amsthm}   % theorems
% \usepackage{amsmath}

\begin{document}
% \bibliographystyle{plainnat} 
% \bibpunct{(}{)}{;}{a}{,}{,}

\begin{center}
Review of \lq\lq THE THERMAL REGIME OF THE CAMPI FLEGREI MAGMATIC SYSTEM
RECONSTRUCTED THROUGH 3D NUMERICAL SIMULATIONS.\rq\rq by
V. Di Renzo et al.
\end{center}

\vspace{1cm}

The manuscript presents a thermal model for the
Campi Flegrei magmatic system in the past $\unit{184}{ka}$. The
results provide the space-time temperature evolution in the Campi
Flegrei caldera region as a consequence of the main caldera-forming
eruptions as well as successive magmatic injections and eruptions, based on
state-of-the-art knowledge of evolution of the volcano
plumbing system. The results obtained are in broad agreement with
temperature data obtained from borehole drills, heat flow measurements
as well as seismic estimations of temperatures for the brittle-ductile
transition. 

The main objectives of the study are clear and the paper is well
organized, though it suffers from frequent repetitions of some
concepts, which is both annoying and misleading. The results obtained
are convincing but discussion on other possible configurations
for the Campi Flegrei plumbing system is lacking, especially in consideration
of the relative
simplicity of the physical model; more
details on the modelling methodology seem to be required, as well as
a more thorough illustration of the results obtained, specifically in
terms of time and space evolution in the three-dimensional
domain. Therefore, I suggest publication only after major revisions.

Major comments:
\begin{enumerate}
  \item The authors claim that their model is able to provide constraints on
    the architecture of the magmatic system at Campi Flegrei (lines 149 - 154;
    lines 503 - 506).
    On the contrary, they seem to use constraints obtained from geophysics and
    petrology
    to design their system and verify that it is one possible solution that
    matches the observed data in terms of measured temperatures. Their claim can
    be justified only if they are able to show that starting from different
    assumptions (e.g., different size/location of small reservoirs after NYT
    erution) they obtain significantly different results.
  \item It would
    be very interesting to see how temperature varies in time, especially as a
    consequence of new magma injections. This is somehow
    shown in Figures 4 and 5, but providing
    time series of temperature at various locations would be much appreciated.
    The time scales for temperatures to vary in the subsurface are not easy to
    grasp form the manuscript if present at all, while they seem to be one of
    the key factors in determining the overall evolution of the system. A
    (possibly 3D) movie of the evolution of temperature in the system
    could significantly improve the manuscript impact.
  \item There is a very long review of previous work and petrological data in
    Section 1.1. I understand that this is the base of the model developed by
    the authors; nonetheless, this part together with other very similar
    descriptions in Sections 3, 4 and 5 makes up most of the manuscript, at the
    expenses of a more detailed description of the model and a much needed
    sensitivity analysis on the space-time distribution of magmatic intrusions.
  \item There is no figure detailing the three-dimensional structure of the
    model, and there is no hint as to how the model looks in the third
    dimension. As the results obtained from 2D modelling are different
    from those obtained in this manuscript, this seems to be quite an important
    point to elucidate.
  \item The authors should describe how Heat3D deals with convection within
    hydrothermal systems, and within magmatic bodies if relevant to their model
    (see below).
  \item The authors claim that model d) in Figure 6, i.e. that obtained by
    considering active hydrothermal convection from the A-MS eruption up to
    $\unit{2500}{ka}$ BP, is a better fit to the measured geothermal gradients.
    This statement needs to be sustained by a measure of the goodness of fit, as
    Figure 6 alone is not enough to support it. Moreover, I don't understand why
    AGIP borehole temperature data from 1987 AD are compared to simulation
    results
    referring to pre-Monte Nuovo (1538 AD). They should be compared to results
    from models a) to d) (different duration of hydrothermal system convection)
    prolonged till after the last eruption. I also wonder how their results
    would differ if no hydrothermal convection is considered at all.
  \item Surface deformation is not considered in the model. The assumption is
    justified but it has to be stated.
  \item Figure 5d: I seem to understand that magma chambers are connected
    through dikes to the deep reservoir; is this not the case here? Why?
\end{enumerate}

A few minor remarks:
\begin{itemize}
  \item The authors do not describe how they deal with shallow magma reservoirs
    emplacement. I suspect it is instantaneous, but it must be stated and
    justified. The authors also need to state whether convection within
    magmatic bodies is
    taken into account or not, and why. Moreover, do smaller eruptions after NYT
    empty the respective magma chambers or not? Why?
  \item At line 110, I suspect the authors mean P-SV velocity conversion when
    they write PSv velocities.
  \item Lines 149 - 154: the authors seem to state that mass and momentum
    conservation equations can be satisfied by a larger domain of solutions
    than the energy conservation equation, which they solve. This statement
    needs to be justified; there is no evidence in the paper to support it.
  \item Lines 250 - 252: it is not clear what are the 100 mesh files cited here.
    How and why are they different? What are the different results obtained?
  \item Lines 263 - 301: I suggest this detailed information on how petrological
    constraints where incorporated into the model is included in the Results
    section (Section 4), where it is largely repeated. At the end of point 8, a
    reference
    to Figure 2 would be beneficial.
  \item Lines 289 - 291: it seems that both the resident phonolitic magma and
    the trachytic one entering the system are at $\unit{6}{\kilo\metre}$ depth.
    I'm not sure this is correct.
  \item I suggest replacing the reference to Longo et al., 2006 with the more
    relevant Montagna
    et al.,
    \lq Time scales of mingling in shallow magmatic reservoirs\rq, 2015 (or
    adding the latter).
  \item Lines 314 - 317: is the initial temperature gradient of
    $\unit{20}{\degreecelsius}$ assumed or is it obtained by letting the deep
    reservoir cool?
  \item Lines 416 - 445: this is pretty much a repetition of what had been
    written in Section 4. Results. I suggest a merge.
  \item Lines 448 - 455: I don't see the point of this discussion on the grid
    location of the boreholes. Figure 8 shows a very large temperature
    difference in a narrow region at depth (e.g. panel a), a
    $\unit{200}{\degreecelsius}$ variation in $\unit{400}{\metre}$). It would
    be interesting to have the boreholes indicated in Figure 7a), to understand
    the reasons behind this large variation.
  \item I suggest reducing the size of Section 6. Conclusions, given also that
    most of its content is a repetition.
  \item Figure 1: the caption should state what the coloured dots are.
  \item Figure 2: axes and their labels are very hard to discern. Need also to
    add units.
  \item Figure 3: the profile is at the center of the caldera, I guess, but this
    should be stated in the caption.
  \item Figures 4 and 5: as for Figure 2, axes and labels are too small to be
    readable and units of measurements are not clear (albeit indicated) -
    increasing font size should suffice. The same holds for the temperature
    colorbar. In Figure 5, what is the meaning of the upward arrow in the
    middle of the domain?
  \item Figure 9: what is the $x$ axis? Horizontal distance from what?
\end{itemize}

I suggest a thorough check for English language, typos and spelling. Some
examples:
\begin{itemize}
  \item line 30: material is uncountable;
  \item line 36 and elsewhere: I have never heard of magma chamber
    refreshing; maybe rejuvenation would be more appropriate;
  \item line 48: studies by instead of studies of;
  \item line 61: volcanic areas;
  \item line 65: This is the case - which case? Rephrase;
  \item line 84: whose instead of which;
  \item line 132: depths larger/greater than;
  \item line 145: different eruptions;
  \item line 163: what does cal stand for?;
  \item line 170: of-depths should be of depths;
  \item line 179: reservoirs;
  \item line 367: on the basis of their results;
  \item line 456 last, not latest;
  \item lines 113 - 115; 121 - 122; 246 - 249: awkward phrasing, hard to
    decipher.
  \item Reference formatting should also be carefully checked, and they must be
    ordered chronologically in the text as per the Guide for Authors (e.g.
    lines 70, 72, 77; lines 624, 640; line 831, link not working).
\end{itemize}

\vspace{1cm}

With best wishes, \\
Chiara P. Montagna, INGV Pisa.

%\bibliography{/home/montagna/TOURO/science/bibliography/biblio} 
\end{document}
