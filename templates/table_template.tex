\documentclass[authoryear,preprint,review,12pt,dvips]{article}

\usepackage{amssymb}
\usepackage{amsmath, amsthm, amssymb}
\usepackage{lscape}
\usepackage{tabularx}
\usepackage[tableposition=top]{caption}
\usepackage{rotating} 
\usepackage{multirow}
\usepackage{booktabs}   % nicer horizontal lines
\usepackage{array}      % needed by tabularx; adjust row height in tables
\usepackage{longtable}  % tables split on multiple pages
\usepackage[cm]{fullpage} % smaller margins
\usepackage{adjustbox}    % center table if it is wider than textwidth

\renewcommand\floatpagefraction{.9}
\renewcommand\topfraction{.9}
\renewcommand\bottomfraction{.9}
\renewcommand\textfraction{.1}   
\setcounter{totalnumber}{50}
\setcounter{topnumber}{50}
\setcounter{bottomnumber}{50}

\captionsetup[table]{labelfont=bf,labelsep=space,singlelinecheck=off}

\begin{document}

\thispagestyle{empty}

\begin{table}[ht]
% substitute table with sidewaystable (needs rotating package) for
% landscape oriented tables. latex+dvipdf to obtain result
\fontsize{3mm}{4mm}\selectfont{
\renewcommand{\tablename}{\fontsize{3mm}{4mm}\selectfont{Table}}
\centering
\caption{\fontsize{3mm}{4mm}\selectfont{Symbols used in this paper
and their units with description.}}
\label{symbols1}
\begin{tabular}{l  l  l}
\hline

Symbol & Unit & Description\\

\hline
$N_v$ & m$^{-3}$ & Bubble number density per unit volume of melt\\

\hline

\end{tabular}
}
\end{table}

\begin{longtable}{c p{0.2\textwidth} p{0.2\textwidth}
    p{0.2\textwidth} % splits on pages, does not work with sidewaystable
  c p{0.2\textwidth}}
\toprule

No & Authors & Institution(s) & Title & O/P & Notes \\

\midrule
40 & A Namiki & U Tokyo & Intermittent and efficient outgassing by
the upward propagation of film ruptures in
a bubbly magma & O & analog modeling \\
\bottomrule

\end{longtable}

\end{document}