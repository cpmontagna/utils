\documentclass[compress,green]{beamer}
\mode<presentation>

%\usepackage{subfigure}
%\usepackage{multicol}
%\usepackage{amsmath}
%\usepackage{epsfig}
%\usepackage{graphicx}
%\usepackage[all,knot]{xy}
%\xyoption{arc}
%\usepackage{url}
\usepackage{multimedia}
\usepackage[italian]{babel}
%\usepackage{movie15}
%\usepackage{hyperref}

\usetheme{Madrid} 
\usecolortheme{seagull}
\useinnertheme{rounded}

\definecolor{Red}{rgb}{1,0,0}
\definecolor{Blue}{rgb}{0,0,1}
%\definecolor{Green}{rgb}{0,1,0}
\definecolor{magenta}{rgb}{1,0,.6}
\definecolor{lightblue}{rgb}{0,.5,1}
\definecolor{lightred}{rgb}{1,0.2,0.2}
%\definecolor{lightpurple}{rgb}{.6,.4,1}
%\definecolor{gold}{rgb}{.6,.5,0}
%\definecolor{orange}{rgb}{1,0.4,0}
%\definecolor{hotpink}{rgb}{1,0,0.5}
%\definecolor{newcolor2}{rgb}{.5,.3,.5}
%\definecolor{newcolor}{rgb}{0,.3,1}
%\definecolor{newcolor3}{rgb}{1,0,.35}
%\definecolor{darkgreen1}{rgb}{0, .35, 0}
%\definecolor{darkgreen}{rgb}{0, .6, 0}
\definecolor{darkred}{rgb}{.75,0,0}
%\xdefinecolor{olive}{cmyk}{0.64,0,0.95,0.4}
%\xdefinecolor{purpleish}{cmyk}{0.75,0.75,0,0}

\setbeamertemplate{footline}{}

\title{\textcolor{darkred}{A forward modeling approach to relate geophysical observables at active volcanoes to deep magma dynamics.}}
\author{\underline{C. P. Montagna}$^1$, A. Longo$^1$, P. Papale$^1$, M. Vassalli$^2$, G. Saccorotti$^1$, A. Cassioli$^3$}
\institute{1. Istituto Nazionale di Geofisica e Vulcanologia, Pisa, Italy\\
2. School of Geological Sciences, University College Dublin, Ireland\\
3. Dipartimento di Sistemi e Informatica, Università di Firenze, Italy}
\date[AGU Fall Meeting 2010]{AGU Fall Meeting 2010 \\ San Francisco \\ December 17, 2010}

\begin{document}

\frame{
  \titlepage
    }

\frame{
\frametitle{Rationale}
\begin{center}
\textcolor{Red}{Mass movements} and associated \textcolor{Red}{stress changes} in volcanic systems translate into variations of geophysical quantities commonly measured at the surface: \\
\uncover<2->{\textcolor{magenta}{seismicity, ground displacement} and \textcolor{magenta}{gravity changes}.}
\uncover<3->{
\begin{block}{Understanding signals at active volcanoes means gaining the capability to \lq\lq look into\rq\rq an active volcano:}
\begin{itemize}
\item improvement of the knowledge on volcano physics;
\item identification of the short-term precursors to eruptions.
\end{itemize}
\end{block}
}
\end{center}
}

\frame{
\frametitle{Method}
\begin{center}
\begin{columns}
\column{0.7\textwidth}
\includegraphics[height=3.75cm]{../media/figures/dynamics.png}
\column{0.3\textwidth}
simulation of magma movements:\\ 
fluid dynamical code \texttt{GALES} {\tiny(Longo et al. 2006)}
\end{columns}
\pause
pressure variations resulting from magma motion are used as synthetic sources of ground displacement
\pause
\begin{columns}
\column{0.6\textwidth}
\includegraphics[height=3cm]{../media/figures/displacement.png}
\column{0.4\textwidth}
elastic wave propagation in rocks: \\
\begin{itemize}
\item analytical solutions in an homogeneous infinite medium
\item elasto-dynamics code \texttt{ELM} {\tiny(O'Brien and Bean 2004)}
\end{itemize}
\end{columns}
\end{center}
}

\frame{
\frametitle{Simulated systems}
\includegraphics[height=0.4\textheight]{../media/figures/cf_all.png}
\pause
\includegraphics[height=0.4\textheight]{../media/figures/e13.png}
\pause
\begin{block}{}
\begin{itemize}
\item mingling magmas
\item no crystals
\item isothermal
\item exsolution law: Papale et al. 2006
\item viscosity: Giordano et al. 2008
\end{itemize}
\end{block}
}

\frame{
\frametitle{Magma dynamics: Campi Flegrei}
\begin{columns}
\column{0.5\textwidth}
\movie{\includegraphics[width=0.9\textwidth]{../media/figures/comp0_all1.png}}{./comp_all1.avi}
\column{0.5\textwidth}
\movie{\includegraphics[width=0.9\textwidth]{../media/figures/comp0_all2.jpg}}{./comp_all2.avi}
\end{columns}
}

\frame{
\frametitle{Pressure variations}
\includegraphics[height=0.9\textheight]{../media/figures/deltap_all1.png}
}

\frame{
\frametitle{Pressure variations - upper chamber}
\begin{columns}
\column{0.7\textwidth}
\textcolor{Blue}{Horizontal} vs \textcolor{Red}{vertical} chamber
\includegraphics[width=\textwidth]{../media/figures/deltap_cf.jpg}
\column{0.3\textwidth}
\includegraphics[height=0.4\textheight]{../media/figures/deltap_all1.jpg}\\
\includegraphics[height=0.4\textheight]{../media/figures/deltap_all2.jpg}
\end{columns}
}

\frame{
\frametitle{Ground deformation}
\begin{columns}
\column{0.5\textwidth}
\begin{center}
\textcolor{Blue}{Horizontal chamber}
\includegraphics[height=0.4\textheight]{../media/figures/cf_all1_disp.png}\\
\includegraphics[height=0.4\textheight]{../media/figures/cf_all1_spectra.png}
\end{center}
\column{0.5\textwidth}
\begin{center}
\textcolor{Red}{Vertical chamber}
\includegraphics[height=0.4\textheight]{../media/figures/cf_all2_disp.jpg}\\
\includegraphics[height=0.4\textheight]{../media/figures/cf_all2_spectra.jpg}
\end{center}
\end{columns}
rich in low frequencies
}

\frame{
\frametitle{Gravity changes}
\begin{columns}
\column{0.5\textwidth}
\begin{center}
\textcolor{Blue}{Horizontal chamber}\\
\includegraphics[height=0.4\textheight]{../media/figures/dens_all1.jpg}\\
\includegraphics[height=0.4\textheight]{../media/figures/gravity_all1.jpg}
\end{center}
\column{0.5\textwidth}
\begin{center}
\textcolor{Red}{Vertical chamber}\\
\includegraphics[height=0.4\textheight]{../media/figures/dens_all2.jpg}\\
\includegraphics[height=0.4\textheight]{../media/figures/gravity_all2.jpg}
\end{center}
\end{columns}
}

\frame{
\frametitle{Magma dynamics: Etna}
\begin{center}
\movie{\includegraphics[totalheight=0.9\textheight]{../media/figures/comp0_etna.png}}{./comp_e13.avi}
\end{center}
}

\frame{
\frametitle{Ground displacement: low frequencies}
\begin{columns}
\column{0.5\textwidth}
\begin{center}
\includegraphics[width=\textwidth]{../media/figures/etna_disp.png}
\vspace{0.5cm}
highest energy content at \\
periods around $100\unit{s}$
\end{center}
\column{0.5\textwidth}
\begin{center}
analytical solution:\\
Green's function integration\\
\vspace{1cm}
\includegraphics[width=\textwidth]{../media/figures/etna_spectra.png}
\end{center}
\end{columns}
}

\frame{
\frametitle{Ground displacement: seismic frequency range}
\begin{columns}
\column{0.5\textwidth}
\vspace{0.5cm}
\begin{center}
maximum displacement in seismic range $\sim 10\unit{\mu m}$ \\
\vspace{1cm}
patterns of rock displacements\\
deformed by heteorgeneities
\end{center}
\column{0.5\textwidth}
\includegraphics[width=\textwidth]{../media/figures/etna_disp_elm.png}
\end{columns}
\includegraphics[width=\textwidth]{../media/figures/etna_snapshots.png}
}

\frame{
\frametitle{Concluding remarks}
\begin{center}
\begin{itemize}
\item new magma entering shallow chambers rapidly mixes up with the resident magma and looses its identity, over the time scale of hours
\item the evolution of pressure can be very complex, with regions of the magmatic system undergoing overall pressure decrease or increase
\item pressure oscillations emerge, having periods from tens of seconds to hours and amplitude decreasing with increasing oscillation frequency
\item ground oscillations with periods of tens to hundreds of seconds (ULP) emerge as diagnostic of magma convection and mixing in magmatic reservoirs
\end{itemize}
\end{center}
\pause
\includegraphics[width=0.9\textwidth]{../media/figures/conclusion.png}
}

\frame{
\begin{center}
\huge{Thank you!}
\end{center}
}

\end{document}

================ schema ===================
1. introduzione generale, dicendo anche che sempre ho magma buoyant che risale!!
2. sistemi simulati: e13 [elm], all1 e all2 [oblate - prolate]
3. differenti depth, size, composition, gas content
4. ULP e tutto quanto

=========== TODO =========================
0. cambiare figure deltap - mel
1. figure giuste spettri e ground disp all2 - mel
