\documentclass{report}
\setlength{\parindent}{0em}

%\usepackage[utf8]{inputenc}
%\usepackage[english]{babel}
%\usepackage{amsmath,amsfonts,amsthm,amssymb}
%\usepackage{graphicx}
%\usepackage{hyperref}
%\usepackage{eurosym}
%\usepackage{booktabs}
%\usepackage{listings}
%
%\usepackage{float}
%\usepackage{titlesec}
%\usepackage[left=3cm, right=2cm]{geometry}
%\usepackage{pdfpages}
%\usepackage{afterpage}
%\usepackage{etoolbox}
%
%\usepackage{titlesec}
%\usepackage{emptypage}
%\usepackage{blindtext}
%
%\usepackage{comment}
%\lstset{language=Matlab,%
%      basicstyle=\small\ttfamily,
%      numbers=left,
%      numberstyle=\tiny,
%      stepnumber=2,
%      frame=lines
%}
%
\pagenumbering{gobble}

\begin{document}

\begin{figure}[h]
  \includegraphics[width=0.15\textwidth]{INGV_LOGO_ACRONIMO.eps}
\end{figure}

\begin{center}
    \Huge \textbf{Personal Statement} 
    \par
    \vspace{2mm}
    \Large Chiara P. Montagna
\end{center}

\par
\vspace{1 cm}



I would like to apply for the position of Topical Editor of the Volcanica journal. In am a mid-career researcher who specializes in the study of magmatic processes through the use of mathematical, physical and numerical modeling; it seems to me that none of the current Volcanica editors has similar expertise, thus I believe my addition could be beneficial. 

I have been interested in Volcanica since its beginning. I believe open science is the way to go, because it enables scientitsts to produce better science and communicate it better. Diamond open access certainly goes in this direction. I want to contribute to the Volcanica community because the journal's vision, to me, is a promising step for a brighter future for scientific publishing. 

\end{document}